\documentclass{beamer}
\usepackage{xltxtra,fontspec,xunicode}
\usepackage{xeCJK} %should be put at the very begining,else compile error,wiered..
\usepackage{tikz}
\usetikzlibrary{shapes,arrows}
\usepackage{graphicx}

\mode<presentation>
{
  \usetheme{Warsaw}
  % or ...

  \setbeamercovered{transparent}
  % or whatever (possibly just delete it)
}


\usepackage[english]{babel}
% or whatever

\usepackage[latin1]{inputenc}
% or whatever

\let\sups\relax	%since tipa confilict with fontenc, rlax it here
\usepackage{times}
\usepackage[T1]{fontenc}
\usepackage{tipa}
% Or whatever. Note that the encoding and the font should match. If T1
% does not look nice, try deleting the line with the fontenc.
\setCJKmainfont{WenQuanYi Micro Hei Mono}
\setCJKmonofont{Hei}

\title[如何解决不确定性] % (optional, use only with long paper titles)
{如何解决不确定性}
\subtitle{之梅花易数、最大熵解法}

\author[shanhongjie] % (optional, use only with lots of authors)
{shanhongjie}
% - Use the \inst{?} command only if the authors have different
%   affiliation.

\date[Short Occasion] % (optional)
{\today}

\subject{不确定性}
% This is only inserted into the PDF information catalog. Can be left
% out. 



% If you have a file called "university-logo-filename.xxx", where xxx
% is a graphic format that can be processed by latex or pdflatex,
% resp., then you can add a logo as follows:

% \pgfdeclareimage[height=0.5cm]{university-logo}{university-logo-filename}
% \logo{\pgfuseimage{university-logo}}



% Delete this, if you do not want the table of contents to pop up at
% the beginning of each subsection:
\AtBeginSubsection[]
{
  \begin{frame}<beamer>{Outline}
    \tableofcontents[currentsection,currentsubsection]
  \end{frame}
}


% If you wish to uncover everything in a step-wise fashion, uncomment
% the following command: 

%\beamerdefaultoverlayspecification{<+->}


\begin{document}

% Define block styles
\tikzstyle{decision} = [diamond, draw, fill=blue!20, 
    text width=4.5em, text badly centered, node distance=3cm, inner sep=0pt]
\tikzstyle{block} = [rectangle, draw, fill=blue!20, 
    text width=7em, text centered, rounded corners, minimum height=4em]
\tikzstyle{line} = [draw, -latex']
\tikzstyle{cloud} = [draw, ellipse,fill=red!20, node distance=3cm,
    minimum height=2em]

\begin{frame}
  \titlepage
\end{frame}

\begin{frame}{Outline}
  \tableofcontents
  % You might wish to add the option [pausesections]
\end{frame}


% Since this a solution template for a generic talk, very little can
% be said about how it should be structured. However, the talk length
% of between 15min and 45min and the theme suggest that you stick to
% the following rules:  

% - Exactly two or three sections (other than the summary).
% - At *most* three subsections per section.
% - Talk about 30s to 2min per frame. So there should be between about
%   15 and 30 frames, all told.


\section{什么是不确定性}

\subsection{什么是不确定性}
\begin{frame}{什么是信息}
  Abstractly, information can be thought of as the resolution of uncertainty.
\end{frame}
\begin{frame}{什么是不确定性}
  Uncertainty is a situation which involves imperfect and/or unknown information.
\\
  不确定的几个要素:主体、时间、客体。比如:
  \begin{itemize}
    \item 张三对李四昨天吃什么不确定。(张三没有接收到该信息。)
    \item 张三对自己明天吃什么不确定。(事情未发生。)
    \item 人类对能否攻克艾滋病不确定。(理论与技术不够。)
  \end{itemize}

\end{frame}

\subsection{怎么度量不确定性}
\begin{frame}{信息量}
  \begin{itemize}
    \item   假设X是一个离散型随机变量,其取值集合为X,概率分布函数为
      $\forall x \in \mathbf{X}:p(x)=Pr(X=x)$,
      我们定义事件$X=x$的信息量为:
  \begin{equation}
    I(x)=\log(\frac{1}{p(x)})
  \end{equation}

    \item 可见,一个事件发生的概率越大,则它所携带的信息量就越小。
  \end{itemize}

\end{frame}
\begin{frame}{熵}
  \begin{itemize}
    \item 对于一个随机变量X而言,它的所有可能取值的信息量的期望$E[I(x)]$就称为熵。
  \begin{equation}
    H(X)=E[I(x)]=\sum_{x \in X} p(x)I(x) 
    = \sum_{x \in X} p(x)\log(\frac{1}{p(x)})
  \end{equation}

    \item 熵,量化了不确定性的大小,因此也量化了解决不确定性事件的意义的大小。
  \end{itemize}
\end{frame}

\subsection{不确定性的哲学观点}
\begin{frame}{不确定性的哲学观点}
  \begin{itemize}
    \item 决定论。又称拉普拉斯信条,认为每个事件的发生,包括人类的认知、举止、决定和行动都是因为先前的事而有原因地发生。如果从原始宇宙以来,有一连串的事件注定地、从未中断地发生,自由意志则是不可能的。
    \item 知识论、信念、近似、概率、不确定性、神秘主义
    \item 不可知论。认为形而上学的一些问题,例如是否有来世、鬼神、天主是否存在等,是不为人知或者根本无法知道的想法或理论。
    \item 虚无主义,为怀疑主义的极致形式。认为世界、生命(特别是人类)的存在是没有客观意义、目的以及可以理解的真相。
  \end{itemize}
\end{frame}
\begin{frame}{不确定性的近代观点总结}
  以下理论是不确定性的依据:
  \begin{itemize}
    \item 量子理论的不确定性原理
    \item 对未来预测的计算在本宇宙中进行,将反过来作用于本宇宙,从而影响预测结果
    \item 混沌理论描述的计算复杂性,存在计算极限
  \end{itemize}
\end{frame}

\section{梅花易数}
\subsection{Hello World} 
  \begin{frame}{简介}
    梅花易数,是中国古代占卜法之一。以《易经》为基础,结合五行卜算吉凶。
  \end{frame}
  \begin{frame}{占卜步骤}
    \begin{itemize}
      \item 确定事件。无事不卜。
      \item 首先尝试理性的解决。不疑不卜。
      \item 起卦:筮草起卦,龟壳起卦,铜钱起卦,意念起卦\ldots
      \item 解卦
    \end{itemize}
  \end{frame}
  \begin{frame}{起卦}
    \begin{itemize}
      \item 任选一种方式起卦,得到本卦。不妨记作100010(5),1为阳,0为阴,括弧里表示发生了爻动。比如硬币起卦,取3枚硬币,每丢一次得到一爻(001->0, 101->1, 000->0(动), \ldots)
      \item 根据本卦234爻和345爻生成互卦000001
      \item 根据本卦及爻动生成之卦100000
      \item 分辨出:体(没有爻动的部分)卦为100,用卦(有爻动的部分)为010
    \end{itemize}
  \end{frame}
  \begin{frame}{解卦}
    \begin{itemize}
      \item 根据本卦100010查询易经卦辞、象辞,并根据爻动查询爻辞。
      \item 根据互卦000001查询易经文本,判断推动事物发展的内部矛盾。
      \item 根据之卦100000查询易经文本,判断十五发展的最终形态。
      \item 查询体卦100和用卦010所属五行的相生相克的关系判定吉凶、损耗、进益。体代表本体、本质,用代表客体、应用。
      \item 综合判断。
    \end{itemize}
  \end{frame}

\subsection{评价}
\begin{frame}{评价}
  \begin{itemize}
    \item 无法重复实验。蒙卦卦辞“匪我求童蒙,童蒙求我。初筮告,再三渎,渎则不告。”
      从体系上拒绝了针对同一事件多次占卜,保护了自身的“自圆其说”。可见是一个封闭的系统。
    \item 取象类比,不具证明效力,更多是一种归纳法则。
    \item 缺乏理论依据,量子纠缠及宇宙全息论等等只是暂时提供了一点立锥之地。
    \item 易经原本的基础上发展了太多的糟粕,增大了“取其精华”的难度。
    \item 在对事件一无所知而一筹莫展时,占卜比随便选择更容易让人接受。
    \item 占卜为泡mm必备基本技能。
    \item 占卜的结果,是一种主观的假设。
  \end{itemize}
\end{frame}
     

\section{最大熵}
\subsection{Hello World}
\begin{frame}{简介}
  \begin{itemize}
    \item 
  最大熵模型:我们的预测要满足全部已知的条件,而不能对未知的情况做任何主观假设。即在遇到不确定性时,要保留所有可能性。
\item
  为了准确的估计随机变量的状态,我们一般习惯性最大化熵,认为在所有可能的概率模型(分布)的集合中,熵最大的模型是最好的模型。
  \end{itemize}
\end{frame}
\begin{frame}{最大熵建模方法}
  \begin{itemize}
    \item 实际问题转换为概率问题
    \item 熵公式, 作为最大化求解目标
    \item 已有的知识转换为等式和不等式作为constraints
  \end{itemize}
\end{frame}

\begin{frame}{资产组合示例}
  假设我2016年的7个品种的资产的最大回撤分别为: 0.15,0.24,0.90,0.33,0.25,0.15,0.17, 回报率分别为0.05,0.1,-0.1,0.03,0.05,0.12,0.18。 \\我要求总的最大回撤率不超过0.3, 总的回报率不低于0.2, \\那么我如何构建资产组合呢?
\end{frame}
\begin{frame}{资产组合示例}
  \begin{itemize}
    \item 资产组合问题转换为每个品种的选中的概率问题。即最终希望得到一组比例:$p_1,p_2,p_3,p_4,p_5,p_6,p_7$
    \item 熵的公式:
  \begin{equation}
    H(X)= \sum_{i=1}^{7} p_i\log(\frac{1}{p_i})
  \end{equation}
\item 建立约束条件:
  \begin{itemize}
    \item $ 0.15*p_1 + 0.24*p_2 + 0.90*p_3 + 0.33*p_4 + 0.25*p_5 + 0.15*p_6 + 0.17*p_7 < 0.3 $
    \item $ 0.2< 0.05*p_1 + 0.1*p_2 + -0.1*p_3 + 0.03*p_4 + 0.05*p_5 + 0.12*p_6 + 0.18*p_7 $
  \end{itemize}
\item 求解。可利用R/Python的优化包解决。

  \end{itemize}
\end{frame}

\subsection{评价}
\begin{frame}{评价}
  \begin{itemize}
    \item 模型本身就允许充分利用已有的知识。
    \item 对未知事物不做任何假设,没有任何偏见。
    \item 熵是一种期望,因此适合长期反复使用。
    \item 模型的形式非常简单。
  \end{itemize}
\end{frame}

\section{Reference}
\subsection{Reference}
\begin{frame}
  \begin{itemize}
    \item https://en.wikipedia.org/wiki/Information\_theory
    \item https://en.wikipedia.org/wiki/Uncertainty
    \item https://zh.wikipedia.org/wiki/知识论
    \item http://blog.csdn.net/rtygbwwwerr/article/details/50778098
    \item 《数学之美》
    \item 《易经》
    \item 《卦的体用分析与五行生克》
  \end{itemize}
\end{frame}

\end{document}
