\documentclass{beamer}
\usepackage{xltxtra,fontspec,xunicode}
\usepackage{xeCJK} %should be put at the very begining,else compile error,wiered..
\usepackage{tikz}
\usetikzlibrary{shapes,arrows}
\usepackage{graphicx}

\mode<presentation>
{
  \usetheme{Warsaw}
  % or ...

  \setbeamercovered{transparent}
  % or whatever (possibly just delete it)
}


\usepackage[english]{babel}
% or whatever

\usepackage[latin1]{inputenc}
% or whatever

\let\sups\relax	%since tipa confilict with fontenc, rlax it here
\usepackage{times}
\usepackage[T1]{fontenc}
\usepackage{tipa}
% Or whatever. Note that the encoding and the font should match. If T1
% does not look nice, try deleting the line with the fontenc.
\setCJKmainfont{WenQuanYi Micro Hei Mono}
\setCJKmonofont{Hei}

\title[What You Think Is What You Get] % (optional, use only with long paper titles)
{Introduction To \TeX}

\author[shanhongjie] % (optional, use only with lots of authors)
{shanhongjie}
% - Use the \inst{?} command only if the authors have different
%   affiliation.

\date[Short Occasion] % (optional)
{\today}

\subject{TeX}
% This is only inserted into the PDF information catalog. Can be left
% out. 



% If you have a file called "university-logo-filename.xxx", where xxx
% is a graphic format that can be processed by latex or pdflatex,
% resp., then you can add a logo as follows:

% \pgfdeclareimage[height=0.5cm]{university-logo}{university-logo-filename}
% \logo{\pgfuseimage{university-logo}}



% Delete this, if you do not want the table of contents to pop up at
% the beginning of each subsection:
\AtBeginSubsection[]
{
  \begin{frame}<beamer>{Outline}
    \tableofcontents[currentsection,currentsubsection]
  \end{frame}
}


% If you wish to uncover everything in a step-wise fashion, uncomment
% the following command: 

%\beamerdefaultoverlayspecification{<+->}


\begin{document}

% Define block styles
\tikzstyle{decision} = [diamond, draw, fill=blue!20, 
    text width=4.5em, text badly centered, node distance=3cm, inner sep=0pt]
\tikzstyle{block} = [rectangle, draw, fill=blue!20, 
    text width=7em, text centered, rounded corners, minimum height=4em]
\tikzstyle{line} = [draw, -latex']
\tikzstyle{cloud} = [draw, ellipse,fill=red!20, node distance=3cm,
    minimum height=2em]

\begin{frame}
  \titlepage
\end{frame}

\begin{frame}{Outline}
  \tableofcontents
  % You might wish to add the option [pausesections]
\end{frame}


% Since this a solution template for a generic talk, very little can
% be said about how it should be structured. However, the talk length
% of between 15min and 45min and the theme suggest that you stick to
% the following rules:  

% - Exactly two or three sections (other than the summary).
% - At *most* three subsections per section.
% - Talk about 30s to 2min per frame. So there should be between about
%   15 and 30 frames, all told.

\section{What is \TeX}
\subsection{What \TeX is}
\begin{frame}{What \TeX is}
  \begin{itemize}
    \item \TeX{} is a typesetting system.
    \item Follows the design philosophy of separating presentation from content.
  \end{itemize}
\end{frame}
\begin{frame}{Author}
\begin{center}
  \includegraphics{knuthnorris.jpg}
\end{center}
\end{frame}
\begin{frame}{How to pronounce \TeX}
  \begin{itemize}
    \item The name \TeX{} is intended by its developer to be \textipa{/"tEx/}, with the final consonant of loch or Bach. 
    \item The letters of the name are meant to represent the capital Greek letters tau, epsilon, and chi, as \TeX{} is an abbreviation of \textipa{tEXVn} τέχνη , Greek for both ``art'' and ``craft'', which is also the root word of technical. 
    \item English speakers often pronounce it \textipa{/"tEk/}, like the first syllable of technical.
  \end{itemize}
\end{frame}
\subsection{Why we need \TeX}
\begin{frame}{Why not Word, PPT?}
  What You See Is What You Get: So heavy your brain load is!
\begin{center}
\begin{tikzpicture}[node distance = 2.5cm, auto]
    % Place nodes
\tiny
    \node [block] (brain) {brain};
    \node [block, right of=brain] (content) {content};
    \node [block, above of=content] (apperance) {apperance};
    \node [block, right of=content] (hand) {hand};
    \node [block, right of=hand] (word) {word/ppt};
    \node [decision, below of=hand, node distance = 2cm] (eye) {eye check};

    % Draw edges
    \path [line] (brain) -- (content);
    \path [line] (brain) -- (apperance);
    \path [line] (content) -- (hand);
    \path [line] (apperance) -- (hand);
    \path [line] (hand) -- (word);
    \path [line] (word) -- (eye);
    \path [line] (eye) -- (brain);
    \path [line] (eye) -- node  {adjust} (hand);
\end{tikzpicture}
\end{center}
\end{frame}
\begin{frame}{What You Think Is What You Get}
\begin{center}
\begin{tikzpicture}[node distance = 2.5cm, auto]
    % Place nodes
\tiny
    \node [block] (brain) {brain};
    \node [block, right of=brain] (hand) {hand};
    \node [block, right of=hand] (word) {pdf};

    % Draw edges
    \path [line] (brain) -- node {心想} (content);
    %\path [line] (brain) -- node {think} (content);
    \path [line] (content) -- (hand);
    \path [line] (hand) -- node {事成} (word);
    %\path [line] (hand) -- node {get} (word);
\end{tikzpicture}
\end{center}
\end{frame}
\begin{frame}{Design for programmer}
  \begin{itemize}
    \item Free.
    \item Developement life cycle.
    \item Collabration.
    \item De facto stand for writing academic papers. 
    \item Focus on content only, not to worry about the appearance.(Since some code dogs are too tulibility.)
    \item Stability: Bug is Wanted!
  \end{itemize}
\end{frame}
\begin{frame}{\TeX stack}
\begin{center}
  \includegraphics[scale=0.4]{tex_stack.png}
\end{center}
\end{frame}

\section{How to develope an article}
\subsection{Developement life cycle}
\begin{frame}{edit}
  Just like coding.
  \begin{itemize}
    \item editor: vim, emacs, notepad, \ldots
    \item IDE: TeXShop, texlive\ldots
  \end{itemize}
\end{frame}
\begin{frame}{compile}
  xelatex <filename.tex>

  Outputs: filename.pdf, filename.log, \ldots
\end{frame}
\subsection{Document code structure}
\begin{frame}{Document code structure}
  Take article.tex(\LaTeX) as example.
\end{frame}

\subsection{How to write an article}
\begin{frame}{How to write an article}
  \begin{itemize}
    \item How to write a formula
    \item  How to write a table
    \item   How to include a graph
    \item   How to write a graph
  \end{itemize}
\end{frame}
\begin{frame}{How to apply specific style}
  Apply specific style: ACM, IEEE, thesis, \ldots
  \begin{itemize}
    \item download the style file
    \item change ``documentclass'' to the specific style simply.
  \end{itemize}
  Take ieee.tex(\LaTeX) as example.
\end{frame}

\section{How to develope slides}
\subsection{How to develope slides}
\begin{frame}{Slides code structure}
  Take intro.tex(\LaTeX) as example.
\end{frame}

\section{Reference}
\subsection{Reference}
\begin{frame}
  \begin{itemize}
    \item https://en.wikipedia.org/wiki/TeX
    \item https://tobi.oetiker.ch/lshort/lshort.pdf
    \item http://www.ctex.org
  \end{itemize}
\end{frame}

\end{document}
