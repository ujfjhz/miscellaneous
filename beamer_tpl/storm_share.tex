\documentclass{beamer}
\usepackage{xltxtra,fontspec,xunicode}
\usepackage{xeCJK} %should be put at the very begining,else compile error,wiered..
\usepackage{tikz}
\usetikzlibrary{shapes,arrows}
\usepackage{graphicx}
\usepackage{listings}

\mode<presentation>
{
  \usetheme{Warsaw}
  % or ...

  \setbeamercovered{transparent}
  % or whatever (possibly just delete it)
}


\usepackage[english]{babel}
% or whatever

\usepackage[latin1]{inputenc}
% or whatever

\let\sups\relax	%since tipa confilict with fontenc, rlax it here
\usepackage{times}
\usepackage[T1]{fontenc}
\usepackage{tipa}
% Or whatever. Note that the encoding and the font should match. If T1
% does not look nice, try deleting the line with the fontenc.
\setCJKmainfont{WenQuanYi Micro Hei Mono}
\setCJKmonofont{Hei}

\title[Storm] % (optional, use only with long paper titles)
{Storm}
\subtitle{Distributed realtime computation}

\author[shanhongjie] % (optional, use only with lots of authors)
{shanhongjie}

\institute[yihaodian] % (optional, but mostly needed)
{
  IT\_Architecture\_SOA\\
  YHD
}
% - Use the \inst{?} command only if the authors have different
%   affiliation.

\date[Short Occasion] % (optional)
{\today}

\subject{Storm}
% This is only inserted into the PDF information catalog. Can be left
% out. 



% If you have a file called "university-logo-filename.xxx", where xxx
% is a graphic format that can be processed by latex or pdflatex,
% resp., then you can add a logo as follows:

% \pgfdeclareimage[height=0.5cm]{university-logo}{university-logo-filename}
% \logo{\pgfuseimage{university-logo}}



% Delete this, if you do not want the table of contents to pop up at
% the beginning of each subsection:
\AtBeginSubsection[]
{
  \begin{frame}<beamer>{Outline}
    \tableofcontents[currentsection,currentsubsection]
  \end{frame}
}


% If you wish to uncover everything in a step-wise fashion, uncomment
% the following command: 

%\beamerdefaultoverlayspecification{<+->}


\begin{document}

% Define block styles
\tikzstyle{decision} = [diamond, draw, fill=blue!20, 
    text width=4.5em, text badly centered, node distance=3cm, inner sep=0pt]
\tikzstyle{block} = [rectangle, draw, fill=blue!20, 
    text width=7em, text centered, rounded corners, minimum height=4em]
\tikzstyle{line} = [draw, -latex']
\tikzstyle{cloud} = [draw, ellipse,fill=red!20, node distance=3cm,
    minimum height=2em]

\begin{frame}
  \titlepage
\end{frame}

\begin{frame}{Outline}
  \tableofcontents
  % You might wish to add the option [pausesections]
\end{frame}


% Since this a solution template for a generic talk, very little can
% be said about how it should be structured. However, the talk length
% of between 15min and 45min and the theme suggest that you stick to
% the following rules:  

% - Exactly two or three sections (other than the summary).
% - At *most* three subsections per section.
% - Talk about 30s to 2min per frame. So there should be between about
%   15 and 30 frames, all told.


\section{What}

\subsection{History}
\begin{frame}{Apache bigdata family}
  \begin{figure}
    \centering
    \includegraphics[scale=0.22]{apache_projects.jpg}
    %\caption{Apache Storm is a free and open source distributed realtime computation system.}
  \end{figure}
\end{frame}
\begin{frame}{Storm for realtime computation}
  \begin{figure}
    \centering
    \includegraphics[scale=0.5]{storm-vs-hadoop.png}
    \caption{batch processing v.s stream processing}
  \end{figure}
\end{frame}
\begin{frame}{Storm history}
  \begin{itemize}
    \item Originally created by Nathan Marz and team at BackType.
    \item Open sourced after being acquired by Twitter.
    \item Graduate to a top-level project (TLP) on 2014-09-17.
  \end{itemize}
\end{frame}
\begin{frame}{Present status}
  \begin{itemize}
    \item Current stable version: v1.1.0. We use v1.0.2.
    \item Important branches:
      \begin{itemize}
	\item master: for v2.0.0 (java-migration, jstorm-merging)
	\item 1.0.x-branch: for stablizing these releases.
      \end{itemize}
  \end{itemize}
\end{frame}

\subsection{Architecture}
\begin{frame}{Logical topology}
  \begin{figure}
    \centering
    \includegraphics[scale=0.22]{storm-flow.png}
    \caption{A topology is a DAG, the logic for a realtime application.}
  \end{figure}
\end{frame}
\begin{frame}{Logical topology}
  \begin{itemize}
    \item Spout, source of streams in a topology\\
      \begin{itemize}
	\item Read data from external source(kafka, jumper, database,\ldots)
	\item Emit streams of tuples into the topology
      \end{itemize}

    \item Bolt, operators that processing tuple in stream.
    \item Stream, unbounded sequence of tuples.
    
  \end{itemize}
\end{frame}
\begin{frame}{Physical topology}
  \begin{figure}
    \centering
    \includegraphics[scale=0.5]{example-parallelism-topology.png}
    \caption{Worker processes, executors and tasks make a running topology}
  \end{figure}
\end{frame}
\begin{frame}{Physical topology}
  \begin{itemize}
    \item Worker, jvm process, running one or more executors,  occupy a slot(abstraction of <cpu,mem>).
    \item Executor, thread in jvm, running one or more spout/bolt tasks.
  \end{itemize}
\end{frame}
\begin{frame}{Architecture}
  \begin{center}
      \includegraphics[scale=0.45]{storm_arch.jpg}
  \end{center}
\end{frame}
\begin{frame}{Deployment in YHD}
  \begin{figure}
    \centering
    \includegraphics[scale=0.36]{storm_deploy_yhd.png}
  \end{figure}
\end{frame}
\begin{frame}{Features}
  \begin{itemize}
    \item fast \\
      over 1,000,000 tuples/second per node
    \item scalable \\
      easy to extend horizentally
    \item fault-tolerant\\
      storm daemons can die safely
    \item data process guarantee\\
      \begin{itemize}
	\item at most once
	\item at least once
	\item exactly once
      \end{itemize}
    \item easy to learn, code and operate
    \item \ldots
  \end{itemize}
\end{frame}

\subsection{Use case}
\begin{frame}{Usage in YHD}
  \begin{figure}
    \centering
    \includegraphics[scale=0.4]{topologies-in-yhd.png}
    \caption{应用:BI, 精准化,流量分析,风控,反爬虫,等等}
  \end{figure}
\end{frame}
\begin{frame}{Common patterns}
  \begin{itemize}
    \item continous computation
    \item realtime analytics
    \item online machine learning
    \item distributed RPC
    \item ETL
    \item \ldots
  \end{itemize}
\end{frame}

\section{How}
\subsection{Hello world}
\begin{frame}{Example: WordCount}
  \begin{figure}
    \centering
    \includegraphics[scale=0.8]{example-wordcount.png}
  \end{figure}
\end{frame}
\begin{frame}{Design your topology}
  \begin{figure}
    \centering
    \includegraphics[scale=0.4]{wordcount-topo-design.png}
  \end{figure}
\end{frame}
\begin{frame}{Implement - topology}
\lstinputlisting[language=Java,basicstyle=\small]{topology.java}
\end{frame}
\begin{frame}{Implement - spout}
\lstinputlisting[language=Java,basicstyle=\small]{spout.java}
\end{frame}
\begin{frame}{Implement - bolt}
\lstinputlisting[language=Java,basicstyle=\small]{bolt.java}
\end{frame}
\begin{frame}{Deploy your topology}
  \begin{itemize}
    \item packaging\\
      recommend using maven-shade-plugin to make a fat jar with full dependencies.
    \item submit\\
      http://stormui.yihaodian.com
  \end{itemize}
\end{frame}

\subsection{Parallelism}
\begin{frame}{Parallelizing WordCount}
  \begin{figure}
    \centering
    \includegraphics[scale=0.4]{wordcount-grouping.png}
  \end{figure}
\end{frame}
\begin{frame}{Topology parallelism}
  \begin{itemize}
    \item Number of workers(processes)\\
      method: setNumWorkers()
    \item Number of executors(threads)\\
      method: setSpout(),setBold()
    \item Number of tasks(by default one task per executor)\\
      method: setNumTasks()
  \end{itemize}
\end{frame}
\begin{frame}{Stream grouping}
  \begin{itemize}
    \item Shuffle grouping\\
      Tuples are randomly distributed across the bolt's tasks.
    \item Fields grouping\\
      Tuples are partitioned by the fields specified in the grouping.
    \item Local or shuffle grouping
    \item others\\
      Partial Key grouping, all grouping, global grouping \ldots
  \end{itemize}
\end{frame}

\subsection{Guaranteeing message processing}
\begin{frame}{At most once}
  \begin{itemize}
    \item By default, storm guarantee message processed at most once, NO reliability.
    \item Data loss may happen.
  \end{itemize}
\end{frame}
\begin{frame}{At least once}
  \begin{itemize}
    \item User should implement at-least-once using mechanism provided by Storm.
    \item Data duplication may happen.
    \item Reference: http://storm.apache.org/releases/1.0.2/Guaranteeing-message-processing.html
  \end{itemize}
  
\end{frame}
\begin{frame}{At least once - Mechanism}
  \begin{figure}
    \centering
    \includegraphics[scale=0.4]{acker.png}
  \end{figure}
\end{frame}
\begin{frame}{At least once - Mechanism}
  Implementing at-least-once by user:
  \begin{itemize}
    \item Spout side
      \begin{itemize}
	\item generate unique id for every tuple
	\item implement ack(), to be called when fully processed
	\item implement fail(), to be called when fail or timeout
      \end{itemize}
    \item Bolt side
      \begin{itemize}
	\item anchor original tuple
	\item ack or fail
      \end{itemize}

  \end{itemize}
\end{frame}
\begin{frame}{At least once - Example}
  \begin{itemize}
    \item Spout\\
      \_collector.emit(new Values(``field1'', ``field2'', 3) , msgId);
    \item Bolt\\
      \_collector.emit(tuple, new Values(word)); \\
      \_collector.ack(tuple);
      (If you use IBasicBolt, anchor and ack are done automatically.)
  \end{itemize}
\end{frame}

\begin{frame}{Exactly once}
  
  \begin{itemize}
    \item Using Storm Trident to archive exactly-once processing.
    \item Storm Trident is a high-level abstraction on top of Storm core.
    \item No data loss and no data duplication.
    \item Reference: http://storm.apache.org/releases/1.0.2/Trident-tutorial.html
  \end{itemize}
\end{frame}

\subsection{Tuning}
\begin{frame}{Visualization - Running status}
  \begin{figure}
    \centering
    \includegraphics[width=\paperwidth,height=0.6\paperheight]{canvas.png}
    \caption{ad\_behavior\_realtime}
  \end{figure}
\end{frame}
\begin{frame}{Visualization - Running status}
  \begin{itemize}
    \item Width of lines between components represents the flow of tuples between components.\\
      format: <streamid>:<throughput in last 10minute>:<throuhput percentage>
    \item A blue component represents the Spout in the topology.
    \item A red component denote a data bottleneck.
    \item A green component indicate it processing within capacity.
  \end{itemize}
\end{frame}
\begin{frame}{Visualization - History status}
  \begin{figure}
    \centering
    \includegraphics[scale=0.29]{storm-metric.png}
    \caption{http://stormui.yihaodian.com/metric}
  \end{figure}
\end{frame}
\begin{frame}{Visualization - Bolt}
  Metrics for bolts(last 1 minute):
  \begin{itemize}
    \item emit-count\\
      collector.emit() count
    \item ack-count\\
      collector.ack() count
    \item transfer-count\\
      transfer count
    \item execute-count\\
      bolt.execute() count
    \item execute-latency\\
      timestamp execute() ends - timestamp execute() begins
    \item process-latency\\
      timestamp ack() called - timestamp execute() begins
  \end{itemize}
\end{frame}
\begin{frame}{Visualization - Spout}
  Metrics for spouts(last 1 minute):
  \begin{itemize}
    \item complete-latency\\
      The average time a tuple tree takes to be completely processed.(0 if no acking)
    \item ack-count\\
      The number of tuple trees that completely processed .(0 if no acking)
    \item fail-count\\
      The number of tuple trees that explicitly failed or timeout.(0 if no acking)
    \item emit-count\\
      collector.emit() count
    \item transfer-count\\
      transfer count
    \item kafkaOffset, kafkaPartition
  \end{itemize}
\end{frame}
\begin{frame}{Visualization - queues}
  \begin{figure}
    \centering
    \includegraphics[width=0.95\paperwidth,height=0.7\paperheight]{storm-queue.png}
    \caption{Queues(internal message buffers) and executors}
  \end{figure}
\end{frame}
\begin{frame}{Visualization - queues}
  Metrics of recieve or send queue:
  \begin{itemize}
    \item receive/sendqueue
      \begin{itemize}
	\item arrival\_rate\_secs\\
	  tuples arrival rate in last 1 minute
	\item capacity\\
	  queue size
	\item write\_pos\\
	  write position in queue
	\item read\_pos\\
	  read position in queue
	\item population\\
	  write\_pos - read\_pos
	\item sojourn\_time\_ms\\
	  (wp - rp) / Math.max(arrivalRateInSecs, 0.00001) * 1000.0
	\item overflow\\
	  size of overflow buffer
      \end{itemize}
  \end{itemize}
\end{frame}

\begin{frame}{Tuning summary}
  \begin{itemize}
    \item Show visualization in Storm UI to display a visual representation of your topology and find out the data bottleneck.
    \item Verify that you have found the topology bottleneck by rewriting execute() method to perform no operations. Then optimize it.
    \item Increase topology.message.timeout.secs(default 30 seconds) for the topology.
    \item Override topology.max.spout.pending(default 1024 in YHD) for the topology.
      \begin{itemize}
	\item Topologies using the storm core API should start with 1024 and slowly decrease the value as necessary.
	\item Topologies using the Trident API should start with a much lower value, between 1 and 5.
      \end{itemize}
    \item Increase the parallelism for the component.
  \end{itemize}
\end{frame}
\begin{frame}{Basic guidelines for configuring parallelism}
  Consider the following questions:
  \begin{itemize}
    \item What are my data source?
    \item At what rate do these data sources deliver messages?
    \item What size are the messages?
    \item What is my slowest data sink?
  \end{itemize}
\end{frame}
\begin{frame}{Best practice}
  \begin{itemize}
    \item worker count < mechine count\\
      Keep unnecessary network transfer to minimum
    \item 1 Acker per worker(default)
  \end{itemize}
\end{frame}

\section{Future}
\subsection{Storm future}
\begin{frame}{Storm future}
  \begin{quote}
    ``The future is already here, it is just not evenly distributed. -- William Gibson''
  \end{quote}
\end{frame}

\begin{frame}{Integration with external systems}

  \begin{itemize}
    \item     Apache Kafka Integration
    \item     Apache HBase Integration
    \item     Apache HDFS Integration
    \item     Apache Hive Integration
    \item     JDBC Integration
    \item     Redis Integration
    \item     Elasticsearch Integration
    \item     Mongodb Integration
  \end{itemize}
  ref: https://github.com/apache/storm/tree/master/external
\end{frame}

\begin{frame}{Other external libraries}
  Layers on top of storm:
  \begin{itemize}
    \item flux \\
      Easily configure and deploy Storm topologies without embedding configuration in your topology code.
    \item sql \\
      The Storm SQL integration allows users to run SQL queries over streaming data in Storm. (sql->trident topology)
  \end{itemize}
  ref: https://github.com/apache/storm/tree/master/external
\end{frame}

\begin{frame}{Evolution}
  实时计算领域的发展过程,也是实时计算项目间军备竞赛过程。Apache Storm, Twitter Heron, Apache Flink等互相学习,共同进化。Storm进化出或进化着如下features:
  \begin{itemize}
    \item 
    \item 
  \end{itemize}
\end{frame}

\begin{frame}{What's next for Storm}
  for 2.0 and later,
  \begin{itemize}
    \item Clojure to Java
    \item Apache Beam integration\\
      for unified API for dealing with bounded/unbounded data(i.e. batch/streaming)
    \item Enhancements, improvements, \ldots
  \end{itemize}
\end{frame}

\subsection{Bring future to YHD}

\begin{frame}{Bring future to YHD}
  Storm As A Service:
  \begin{itemize}
    \item Visualization
    \item Multi-tenant resource management
    \item Automate application management
  \end{itemize}
\end{frame}

\begin{frame}{Reference}
  \begin{itemize}
    \item http://storm.apache.org/releases/1.0.2/index.html
    \item https://github.com/apache/storm
    \item http://stormui.yihaodian.com/metric
    \item http://wiki.yihaodian.cn/mediawiki/index.php/Storm测试集群
    \item http://wiki.yihaodian.cn/mediawiki/index.php/Storm线上集群
    \item http://wiki.yihaodian.cn/mediawiki/index.php/架构组-流式计算
  \end{itemize}
\end{frame}

\subsection{Q \& A}
\begin{frame}{Q \& A}
  Thank you!
\end{frame}

\end{document}
